\documentclass[11pt,a4paper]{article}
\usepackage{pdflscape}




% Layout + spacing
\usepackage[a4paper,total={6.2in,8.5in}]{geometry}
\usepackage{setspace}
\setstretch{1.2}

% Citations (natbib)
\usepackage[round,sort&compress]{natbib}
\setcitestyle{authoryear,open={(},close={)},citesep={,},aysep={,},yysep={,}}

% Math packages
\usepackage{amsmath}
\usepackage{amssymb}
\usepackage{amsfonts}


% Graphics package
\usepackage{graphicx}

% Float package for table positioning
\usepackage{float}

% Package for creating boxes
\usepackage{mdframed}

% Links (load AFTER apacite)
\usepackage[dvipsnames]{xcolor}
\usepackage{hyperref}
\hypersetup{colorlinks=true, linkcolor=NavyBlue, citecolor=NavyBlue, urlcolor=NavyBlue}

% --- Title control (needs titling) ---
\usepackage{titling}

% Center the title vertically and horizontally
\renewcommand\maketitlehooka{\null\mbox{}\vfill}
\renewcommand\maketitlehookd{\vfill\null}

% Center the whole block; make date sit right below the title
\pretitle{\begin{center}\Large\bfseries}
\posttitle{\par\end{center}}                        % no extra vertical space here
\preauthor{\begin{center}\small}  % makes the author smaller
\postauthor{\end{center}}               % remove author line space
\predate{\begin{center}\par\vspace{0.25em}\normalsize} % tiny gap before date
\postdate{\par\end{center}} % close center

\title{DIARIO \\ OECD ERC Survey}
\author{}            % keep empty
\date{\today}        % will appear right under the title

\begin{document}

\maketitle
\thispagestyle{empty}

\newpage





\newpage
\tableofcontents
\thispagestyle{empty}

\newpage
\setcounter{page}{1}
\section{Suggestions and Ideas}

\newpage


\section{Incidence by Type of Worker}

\begin{figure}[H]
\centering
\includegraphics[width=1.0\textwidth]{Images/ncc_by_worker_type.png}
\caption{Non-compete clauses by worker type}
\label{fig:ncc_by_worker_type}
\end{figure}

 Considering the countries with the larger mismatch between worker and firm's incidence, for \textbf{Japan}, \textbf{Korea}, \textbf{New Zealand}, and \textbf{Poland} it seems that the mismatch is mainly driven by differences in the worker category: ``Other'' (i.e. receptionist, clerk, secretary, waiter, security guard, electrician, plant operator, cleaner, etc.). For \textbf{France} and \textbf{Switzerland}, instead, the mismatch is proportional across types of workers.


\section{Validity of Noncompete clauses}




We want to explore whether there is some interesting correlation between the mismatch betweeen worker and employer incidence values, and the validity of the Noncompete clauses reported. By exploiting an elaboration from OECD on the legislative framework of the countries of our dataset, we seek to create one dummy variable for each country, indicating whether the clause is legal or not(or whether it is ambiguous). The first step is to lie down, if available, relevant criteria that can be cross-checked within out dataset, in order to create such dummies. In the following list we report, for each country, the information that we have on the legal framework, and we proposed the parameters that we can use to create the dummies.

\subsection{Literature}

\cite{starr_behavioral_2020} study how unenforceable contractual provisions may affect behaviour. Even in states that
do not enforce noncompetes, employees who agree to be bound by such provisions show generally
lower mobility levels and evince redirection toward noncompetitors. Decomposing mobility into job offer generation and acceptance, the study detected no evidence of differences in job search, recruitment, or offer activity associated with non-competes. Rather, the effect occurs at the job offer acceptance stage—workers turn down opportunities they otherwise would have accepted.

\cite{prescott_subjective_2024} show that an information treatment that roughly simulates aneducational campaign can cause employees to update their beliefs aboutenforceability—especially when noncompetes are actually unenforceable.After receiving information, employees with an unenforceable noncom-pete report that their noncompete would be less of a factor in choosingwhether to accept employment with a competitor than they do when theymistakenly believe ex ante that their noncompete is enforceable. However,employees as a group do not fully adjust their mobility intentions (that is,they do not report that their noncompete would no longer matter). In fact,a nontrivial fraction of employees who see their noncompete as unenforce-able and who view a lawsuit as unlikely continue to consider their non-compete to be a factor in deciding whether to take a job offer at a compet-itor. This result suggests that moral, reputational, and perhaps financialcosts remain for violating even entirely unenforceable contract provisions.

\cite{boeri_non-compete_2023} find that unenforceable clauses, i.e. clauses that are more likely to be used
just to deter workers from moving and not to protect business interests, go hand-in-hand with lower
wages compared to likely enforceable ones. On the opposite, they are not associated to higher or lower
training opportunities.


\newpage
\subsection{Unenforceability, by Country}

	\textbf{Mexico}

Always unenforceable.

	\textbf{Belgium}

Unenforceable if:
\begin{itemize}
	\item Duration above 12 months (or no duration)
	\item Compensation below 50\% of salary (or no compensation)
	\item No geographic or sectoral scope
	\item Annual gross salary below 43,106 (as of January 2025)
	\item If the salary is between EUR 43,106 and 86,212, a noncompete is unenforceable unless CBA says otherwise
\end{itemize}

	\textbf{Germany}

Unenforceable if:
\begin{itemize}
	\item Duration above 24 months (or no duration)
	\item Compensation below 50\% of salary (or no compensation)
	\item No geographic or sectoral scope
\end{itemize}

	\textbf{Sweden}

Unenforceable if:
\begin{itemize}
	\item Duration above 9/18 months (or no duration) --- but can last longer under special circumstances
	\item Compensation below 60\% of salary (or no compensation, but lower amount is admitted if the employee earns a salary from non-competing work)
	\item No geographic or sectoral scope
\end{itemize}

	\textbf{Canada}

Unenforceable if:
\begin{itemize}
	\item No duration
	\item No geographic or sectoral scope
\end{itemize}
Note: specific rules for Ontario (noncompetes banned in 2021 + sanctions)

	\textbf{Portugal}

Unenforceable if:
\begin{itemize}
	\item Duration above 24 months (or no duration)
	\item No compensation
	\item No geographic or sectoral scope
\end{itemize}

	\textbf{France}

Unenforceable if:
\begin{itemize}
	\item No duration
	\item No compensation
	\item No geographic or sectoral scope
\end{itemize}

	\textbf{Spain}

Unenforceable if:
\begin{itemize}
	\item Duration above 24 months for technical employees and 6 months for other workers (or no duration)
	\item No compensation
	\item No geographic or sectoral scope
\end{itemize}

	\textbf{Switzerland}

Unenforceable if:
\begin{itemize}
	\item Duration above 36 months (or no duration)
	\item No geographic or sectoral scope
\end{itemize}

	\textbf{Poland}

Unenforceable if:
\begin{itemize}
	\item No duration
	\item Compensation below 25\% of salary (or no compensation)
	\item No geographic or sectoral scope
\end{itemize}

	\textbf{Japan}

Unenforceable if:
\begin{itemize}
	\item No duration
	\item No geographic or sectoral scope
\end{itemize}

	\textbf{United Kingdom}

Unenforceable if:
\begin{itemize}
	\item No duration
	\item No geographic or sectoral scope
\end{itemize}

	\textbf{New Zealand}

Unenforceable if:
\begin{itemize}
	\item No duration
	\item No geographic or sectoral scope
\end{itemize}

	\textbf{Korea}

Unenforceable if:
\begin{itemize}
	\item No duration
	\item No geographic or sectoral scope
\end{itemize}
\thispagestyle{empty}

\newpage
\bibliographystyle{plainnat}
\bibliography{OECD_ERC_Survey}

\end{document}
